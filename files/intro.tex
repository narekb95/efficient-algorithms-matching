Maximum matching is a well studied optimization problem on graphs in the theory of computer science. It was first introduced through assignment problem, which is given a set of $A$ of workers and $B$ of machines of the same size and a function $f:A\times B \rightarrow \mathbb{R}^{+}$ that assigns a value of efficiency for each worker on each machine, the problem seeks an injective assignment of workers to machines such that the total efficiency is maximized.

In this report, we describe the most common ideas and approaches to solve the problem and their efficient algorithms. We presented them at the Efficient Algorithms and Data Structures Seminar at HU Berlin during summer semester 2019.

The problem has many different variations. Intuitively, in a given a graph, the goal is to pack pairs of adjacent vertices. In the weighted version of the problem we assign a weight function to each edge and try to find a packing with maximum weight. In this report we are only interested in the unweighted part, where we try to maximize the number of pairs in the packing.

\subsection{Notation}
We notate $\Delta$  to the symmetric difference of two sets, meaning for two sets $A, B$ over the same universe we define $A\Delta B := \left( A \setminus B \right) \cup \left( B \setminus A \right)$.

For a vertex set $S$ we notate by $N(S)$ to the neighborhood of $S$, which is the set of vertices adjacent to vertices in $S$. Formally $N(S) := \{v: \exists u \in S s.t. \{u, v\} \in E\}$

Some section-specific notations will be defined later in the corresponding sections.
